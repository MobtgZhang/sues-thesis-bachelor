\section{交叉引用、链接}
\subsection{交叉引用}
这里的交叉引用指的是,需要再response.tex 中引用revision.tex的某一段文本,可以通过以下步骤实现:

\begin{enumerate}[label=\arabic*]
    \item 在 revision.tex 中,将需要引用的文本放在 \%<*tag> 和 \%</tag> 之间,即:
    \begin{lstlisting}
        %<*tag>
        Here is the text to be cited.
        %</tag>
    \end{lstlisting}
    \item 在 response.tex 中,利用 \verb|\ExecuteMetaData| 命令引用文本,

    即:\verb|\ExecuteMetaData[revision]{tag}|

    举个例子,正如我们在\cref{subsec:algorithms}中所说的“\ExecuteMetaData[chapters/ch03]{tag}”一样。
\end{enumerate}
\subsection{链接和其他引用}

有一种在论文中脚注的方式进行引用,例如这里\footnote{这是一个脚注}。另外再说明一下链接的使用方法,例如本项目更新地址在我的\href{https://github.com/mobtgzhang/sues-thesis-bachelor}{GitHub},
当然可以将其放在脚注的地方也是不错的选择,这里需要注意的是脚注需要换行,文档模板中也定义了这些内容\footnote{\url{https://github.com/mobtgzhang/sues-thesis-bachelor}}。
