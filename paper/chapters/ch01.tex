\section{绪论}
绪论这篇章主要讲述的是\LaTeX{}的基本介绍,通过简短的介绍,让读者对\LaTeX{}有一个初步的了解,为后面的学习打下基础。
\subsection{\TeX{}}
\TeX{}是一种由美国计算机教授高德纳(Donald Ervin Knuth)编写的排版软件,主要用于生成高质量的结构化文档(尤其是科学、数学和技术文档),它也被用于生成中等规模的书籍。
它因其在生成复杂数学公式方面的强大功能而广泛用于排版学术期刊。
它也被用于生成中等规模的书籍,如《The \TeX{}book》和《The \LaTeX{} Companion》。
它是自由软件,也是一种自由软件标准——任何人都可以自由地使用和改变它的源代码,但是在修改后的代码上仍然必须保留其发布时的版权。
高德纳教授在1989年宣布不再对\TeX{}作任何修改,因此\TeX{}的当前版本是3.1415926,而不是3.0。

\TeX{} 的发音为“tech”,/ˈtɛx/(英语中的“ch”发音为/tʃ/,如英语中的“church”)。\TeX{}的拼写与希腊语中$\tau\epsilon\chi$的拼写相同,意思是“艺术”或“技术”。

\subsection{\LaTeX{}}
\LaTeX{} 是一种使用\TeX{}排版系统的格式,可以粗略地理解为\TeX{}的一种宏包,它提供了一系列的命令和环境,用于简化\TeX{}的使用。
\LaTeX{} 最初设计的目标是为了方便科技论文的编写,而不是书籍的排版,但是随着\LaTeX{}的发展,它已经被广泛用于排版各种类型的文档,包括书籍、报告、简历等等。
\LaTeXe{} 是\LaTeX{}的最新版本,它在\LaTeX{}2.09的基础上进行了改进,提供了更多的功能和更好的稳定性。

\subsection{\LaTeX{}的优缺点}
经常有人喜欢将\LaTeX{}与Word进行比较,认为\LaTeX{}比Word更加强大,但是也有人认为\LaTeX{}比Word更加麻烦。
其实这种对比是没有意义的,因为\LaTeX{}和Word是两种不同的排版系统,各有各的优缺点,我们应该根据自己的需求来选择合适的排版系统。
因为其设计的目标不一致,一种是以“所见即所得”为目标的Word排版软件,例如Microsoft Word、WPS、LibreOffice、openOffice等等,另一种是以“内容与格式分离”为目标的\LaTeX{}排版系统等,所以两者的优缺点也不一致。

不过这里也总结了一些\LaTeX{}的优缺点,供大家参考:
\begin{itemize}
    \item 具有专业的排版能力,尤其是对于数学公式的排版,\LaTeX{}的效果更加出色。
    \item \LaTeX{}的源文件是纯文本文件,所以可以很方便地进行版本控制,例如使用git进行版本控制。
    \item 很容易生成交叉引用,例如图表、公式、参考文献等等。
    \item \LaTeX{} 和\TeX{} 都是跨平台、开源、免费的软件,可以在各种操作系统上使用。
    在本文中,\LaTeX{}的开发环境是在国产操作系统Deepin Linux 20.9上搭建的,但是在Windows、MacOS和FreeBSD上也可以很方便地搭建\LaTeX{}的开发环境。
    \item \LaTeX{}的学习曲线比较陡峭,需要一定的学习成本。
    \item \LaTeX{}的排版效果不是所见即所得,需要编译才能看到最终的效果,所以排查错误也比较麻烦,有时候的错误甚至很难以理解。
\end{itemize}

\subsection{\LaTeX{}和\TeX{}相关的术语和概念}
有几个基本的概念需要在这里说明一下,这些概念在后面的学习中会经常用到,所以在这里先做一个简单的介绍。
\begin{description}
    \item[引擎] \TeX{}的引擎是指\TeX{}的编译器,例如\TeX{}、pdf\TeX{}、Xe\TeX{}、Lua\TeX{}等等,它们的功能都是将\TeX{}源文件编译成PDF文件,有时候可以将它们简单地理解为\TeX{}的不同版本,称为编译器。
    \item[格式] \LaTeX{}的格式是指定义了一组命令的代码集合。
    \item[编译命令] 编译命令是指将\TeX{}源文件编译成PDF文件的命令,例如\texttt{latex}、\texttt{pdflatex}、\texttt{xelatex}、\texttt{lualatex}等等。  
\end{description}

常见的几个基本的引擎有以下几个,通过下面的表格可以看到它们的区别:
\begin{table}[htbp]
    \centering
    \caption{常见的几个引擎}
    \begin{tabular}{llll}
        \toprule
        引擎 & 作者 & 发布时间 & 说明 \\
        \midrule
        \TeX{} & Donald Knuth & 1978 & 最初的\TeX{}引擎 \\
        pdf\TeX{} & Han The Thanh & 1997 & 增加了对PDF的支持 \\
        Xe\TeX{} & Jonathan Kew & 2004 & 增加了对Unicode的支持 \\
        Lua\TeX{} & Hartmut Henkel & 2007 & 增加了对Lua的支持 \\
        \bottomrule
    \end{tabular}
\end{table}
\subsection{可以获取到的\LaTeX{}资源和学习资料}
\LaTeX{}的学习资料非常丰富,可以通过以下几种方式获取到:
\begin{itemize}
    \item \href{https://www.latexstudio.net/}{latexstudio}:这是一个国内的\LaTeX{}学习网站,上面有很多\LaTeX{}的学习资料,例如\LaTeX{}入门教程、\LaTeX{}常见问题解答、\LaTeX{}视频教程等等。
    \item \href{https://tex.stackexchange.com}{tex.stackexchange.com}:这是一个国外的\LaTeX{}问答网站,上面有很多\LaTeX{}相关的问题和解答,可以通过搜索引擎搜索到很多\LaTeX{}相关的问题和解答。
    \item \href{https://www.bilibili.com}{bilibili大学}:这是一个国内年轻人的网站,当然B站是一所大学啊,资源很丰富的!
    \item \href{https://logic.pku.edu.cn/docs/20220116210822631003.pdf}{一份(不太)简短的\LaTeXe{}介绍}:这是一份非常好的\LaTeX{}入门资料,可以通过这份资料快速入门\LaTeX{}。
\end{itemize}
