\section{第三章}
\subsection{模型建立前的准备}
机械零部件三维模型的构建是有限元仿真的基础,这是因为仿真的实现必须从三维的建立开始。三维几何模型的建立一般需要花费工程技术人员大量的时间和精力,工程上三维几何模型的构建非常严格,必须理解零件、部件的形状及相对位置关系,严格按照实际的尺寸进行建模。只有这样才能达到仿真时对可信度的要求同时,还得考虑在虚拟环境下实体模型构建的难易程度.一些对仿真结果影响不大的特征和构建难度比较大但对实际仿真影响不大的特征可以进行简化。根据轨道车辆过道门的结构参数,综合考虑UG的功能、工作量等因素,选取一个较合理的建模方案,对系统进行结构分析。
现对过道门模型进行如下简化处理:

\begin{enumerate}[label=(\arabic*)]
    \item 可以忽略的部位。
    
    \ding{172} 尺寸较小零部件。

    \ding{173} 过道门上控制零部件。
    
    \item 门窗由塑料组成,原本为百叶窗,由于模态分析时不能划分网格,因此改装为一块合金板,与门框固定起来。
\end{enumerate}

轨道车辆过道门的基本参数见表3-1所示

\begin{table}[hp]
    \centering
    \begin{tabular}{cc}
        \toprule[1.5pt]
        项目               & 参数/m  \\
        \midrule[1pt]
        门框宽              & 1.690 \\
        门框高              & 2.155 \\
        \textbf{\#水平净开度} & 1.400 \\
        垂直净开度            & 1.950 \\
        \bottomrule[1.5pt]
    \end{tabular}
\end{table}

\subsection{小结}

本篇介绍了动力学模态分析的基本求解和建模过程,建立轨道车辆过道门有限元模型,合理的确定约束条件,选定了单元类型。并对各个部分进行材料设置并布尔求和,最后通过网格划分,得到轨道车辆过道门的有限元模型,为进一步对模型进行模态分析打下基础。
