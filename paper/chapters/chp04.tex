\section{轨道车辆过道门模态分析}
\subsection{模态分析理论}

模态分析的理论是在机械阻抗与导纳的概念上发展起来的。近十余年来,模态分析理论吸取了振动理论、信号分析、数据处理、数理统计以及自动控制理论的知识,形成了一套独特的理论,它已经成为近年来应用于结构动力学研究的重要方法。

模态分析的基本原理是:将线性定常系统振动微分方程组中的物理坐标变换为模态坐标,使方程组解耦,成为一组以模态坐标及模态参数描述的独立方程,以便求出系统的模态参数。坐标的变换矩阵为模态矩阵,其每一列为模态振型。由振动理论,系统任一点的响应均可表示为各阶模态响应的线性组合。因而,通过求出的各阶模态参数就可以得到任意激励下任意位置处的系统响应。模态分析的最终目标是识别出系统的模态参数,为结构系统的振动特性分析、振动故障诊断和预报以及结构动力学特性的优化设计提供依据。工程中较复杂的振动问题多为像机床主轴箱这样的多自由度系统。对于多自由度系统利用矩阵分析方法,N自由度线性定常系统的运动微分方程为:

\begin{eqnarray}
    M\ddot{X}+C\dot{X}+KX=F
\end{eqnarray}

其中,$M,C,K$分别表示系统的质量、阻尼和刚度矩阵(均为$N\times{N}$阶矩阵),$X,F$表示系统各点位置上的位移响应和激励力向量。

\begin{eqnarray}
    X=\left\{
        \begin{array}{c}
            x_{1}\\
            x_{2}\\
            \dots\\
            x_{n}\\
        \end{array}
    \right\},
    F=\left\{
        \begin{array}{c}
            f_{1}\\
            f_{2}\\
            \dots\\
            f_{n}\\
        \end{array}
    \right\}
\end{eqnarray}